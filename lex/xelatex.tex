\documentclass{article}
\usepackage{ctex}

% PDF书签配置
\usepackage[pdfusetitle]{hyperref}
\hypersetup{unicode}

% 字体
\usepackage{fontspec}

% 页边距
\usepackage{geometry}
\geometry{left=2.5cm,right=2.5cm,top=2.5cm,bottom=2.5cm}

% 首段缩进
\usepackage{indentfirst}

% 表格
\usepackage[table]{xcolor}

% 代码块
\usepackage{listings}
\lstset{
    basicstyle          =   \sffamily,          % 基本代码风格
    keywordstyle        =   \bfseries,          % 关键字风格
    commentstyle        =   \rmfamily\itshape,  % 注释的风格,斜体
    stringstyle         =   \ttfamily,  % 字符串风格
    flexiblecolumns     =   true,       % 不随便添加空格,只在已经有空格的地方添加空格
    breaklines          =   true,   % 自动换行
    numbers             =   left,   % 行号的位置在左边
    showspaces          =   false,  % 是否显示空格,显示了有点乱,所以不现实了
    numberstyle         =   \zihao{-5}\ttfamily,    % 行号的样式,小五号,tt等宽字体
    showstringspaces    =   false,         % 不显示空格
    captionpos          =   t,             % 这段代码的名字所呈现的位置,t指的是top上面
    frame               =   leftline,      % 框架左边竖线
    extendedchars       =   false,         % 解决代码跨页
    lineskip            =   -0.5ex,        % 行间距,ex表示字母的高度
}

% 标题、作者及日期
\title{标题及目录}
\author{张三丰}
\date{\today}

% 正文
\begin{document}
    \maketitle
    \tableofcontents
    \newpage

    % 1
    \section{摘要}
    车牌识别系统(License Plate Recognition 简称LPR)技术基于数字图像处理,是智
    能交通系统中的关键技术,同时他的发展也十分迅速,已经逐渐融入到我们的现实生活
    中。

    关键字:车牌识别系统;图像预处理;字符识别;Matlab;去噪;二值化

    % 2
    \section{引言}
    现在获取用户的方式有很多,很多企业都设置了多个流量入口,将用户导流到微信个人
    号,再通过微信个人号转移到相关的社群。

    \begin{itemize}
        \item{\bfseries kmp算法实现:}
    \end{itemize}

    \begin{lstlisting}[language=Python]
    class Solution:
        def strStr(self, haystack: str, needle: str) -> int:
            if len(needle) == 0:
                return 0
            next_table = self.kmp_next(needle)
            i = 0
            j = 0
            ret = -1
            while i < len(haystack):
                if haystack[i] == needle[j]:
                    if j == len(needle)-1:
                        ret = i - j
                        break
                    i += 1
                    j += 1
                else:
                    if j > 0:
                        i = i - next_table[j-1]
                    else:
                        i += 1
                    j = 0
            return ret

        def kmp_next(self, needle: str) -> List[int]:
            """ 计算kmp next数组 """
            next_table = [0] * len(needle)
            i = 0
            j = 1
            while j < len(needle):
                if needle[i] == needle[j]:
                    next_table[j] = i + 1
                    i += 1
                    j += 1
                else:
                    if i == 0:
                        j += 1
                    else:
                        i = next_table[i-1]
            return next_table


    \end{lstlisting}


    零售企业在用户获取方面一个是可以从自己的消费者着手,付费用户是你的精准用户。
    此外还可以通过应当向用户传递有吸引力的信息,用户加你的群,可以获取什么样的福
    利和优惠。

    % 3
    \section{实验方法}

    % 3.1
    \subsection{数据}
    每个社群都有其生命周期,一个不活跃的社群很难为商家带来足够的利益,所以在
    社群建立以后,促活就是社群运营的一个核心内容了。

    \begin{tabular}{|l|l|l|l|}
        \hline
        \rowcolor[HTML]{EFEFEF}
        姓名 & 年龄 & 性别 & 班级     \\ \hline
        张三 & 29 & 男  & class1 \\ \hline
        李四 & 30 & 女  & class2 \\ \hline
        王五 & 30 & 男  & class3 \\ \hline
    \end{tabular}

    例如新用户入群有个黄金时间,很多用户在入群一段时间后都会把社群屏蔽掉。如
    何给用户用下一个足够深的印象,打造社群的仪式感和足够的入群欢迎仪式是一个
    很好的办法。

    % 3.2
    \subsection{图表}
    用户获取的快,流失的也就快。如果在第一步用户的获取方面不够精准,在留存方
    面就会流失一大部分了。社群在留存方面有天然的优势,会比一般的产品要更高一
    些。而通过提高用户活跃的运营手段,也可以有效的提高用户的黏性,提高用户的
    留存率。

    但是留存率也不是唯一的指标,特别是在社群裂变以后,会进入大量的非精准用户,
    这些用户很难再进入到下一步变现阶段。

    % 3.2.1
    \subsubsection{实验条件}
    企业都是逐利的,绝大多数的企业关心的就是收入和用户新增,没有持续的盈
    利,企业很难存活。所以变现这一步是很多企业关注的要点。

    \begin{itemize}
        \item{\bfseries Golang语言示例:}
    \end{itemize}

    \begin{lstlisting}[language=Go]
    package main

    import "fmt"

    // main 主函数
    func main() {
        fmt.Println("vim-go")
    }
    \end{lstlisting}

    但是需要注意的是一个是广告不能够太过硬,太过频繁,可以与提高活跃的运
    营手段相结合,例如通过优惠信息、秒杀活动或者其他的一些购物活动等方式
    实现用户变现。

    % 3.2.2
    \subsubsection{实验过程}
    用户自传播又称作裂变传播,是整个AARRR模型的最后一环。实现用户的裂变
    传播,才能带来用户的低成本增长,实现用户群体的爆发式增长。

    如果社群的内容或者活动本身就具备可玩性,或者用户的某种需求点被触动,
    用户也会自发的去将内容传播到自己的社交圈子,带来新的用户。而达成这一
    点,社群的运营也就形成了一个闭环。

    % 4
    \section{实验结果}
    社群运营不但能够解决目标的流量瓶颈问题,还能够对用户进行深度运营。建立一个良
    好的高价值的群,是商家盈利的重要因素,每个社群成员都有裂变出更多客户的可能,
    而建立完善的用户运营模型,能够有效的管理社群成员,实现用户的裂变增长。

    % 5
    \section{结论}

    % 6
    \section{致谢}

\clearpage
\end{document}
