\documentclass[a4paper,10pt]{article}

\usepackage[utf8]{inputenc}

% 中文包
\usepackage{CJKutf8}
\usepackage[encapsulated]{CJK}

% PDF书签配置
\usepackage[pdfusetitle]{hyperref}
\hypersetup{unicode}

% 首段缩进
\usepackage{indentfirst}

% 标题、作者及日期
\title{标题及目录}
\author{张三丰}
\date{\today}

% 正文
\begin{document}
\begin{CJK}{UTF8}{gbsn}
    \maketitle
    \tableofcontents
    \newpage

    \begin{tabular}{| l | c | c | c || r |}
    \hline
        名字 & 语文 & 数学 & 英语 & 备注\\
    \hline
        张三 & 87 & 100 & 93 & 优秀 \\
    \hline
        李四 & 75 & 64 & 52 & 补考另行通知\\
    \hline
    \hline
        王五 & 80 & 82 & 78 & 良好\\
    \hline
    \end{tabular}

    % 1
    \section{摘要}
    车牌识别系统(License Plate Recognition 简称LPR)技术基于数字图像处理,是智
    能交通系统中的关键技术,同时他的发展也十分迅速,已经逐渐融入到我们的现实生活
    中。

    关键字:车牌识别系统;图像预处理;字符识别;Matlab;去噪;二值化

    % 2
    \section{引言}
    现在获取用户的方式有很多,很多企业都设置了多个流量入口,将用户导流到微信个人
    号,再通过微信个人号转移到相关的社群。

    零售企业在用户获取方面一个是可以从自己的消费者着手,付费用户是你的精准用户。
    此外还可以通过应当向用户传递有吸引力的信息,用户加你的群,可以获取什么样的福
    利和优惠。

    % 3
    \section{实验方法}

        % 3.1
        \subsection{数据}
        每个社群都有其生命周期,一个不活跃的社群很难为商家带来足够的利益,所以在
        社群建立以后,促活就是社群运营的一个核心内容了。

        例如新用户入群有个黄金时间,很多用户在入群一段时间后都会把社群屏蔽掉。如
        何给用户用下一个足够深的印象,打造社群的仪式感和足够的入群欢迎仪式是一个
        很好的办法。

        % 3.2
        \subsection{图表}
        用户获取的快,流失的也就快。如果在第一步用户的获取方面不够精准,在留存方
        面就会流失一大部分了。社群在留存方面有天然的优势,会比一般的产品要更高一
        些。而通过提高用户活跃的运营手段,也可以有效的提高用户的黏性,提高用户的
        留存率。

        但是留存率也不是唯一的指标,特别是在社群裂变以后,会进入大量的非精准用户,
        这些用户很难再进入到下一步变现阶段。

            % 3.2.1
            \subsubsection{实验条件}
            企业都是逐利的,绝大多数的企业关心的就是收入和用户新增,没有持续的盈
            利,企业很难存活。所以变现这一步是很多企业关注的要点。

            但是需要注意的是一个是广告不能够太过硬,太过频繁,可以与提高活跃的运
            营手段相结合,例如通过优惠信息、秒杀活动或者其他的一些购物活动等方式
            实现用户变现。

            % 3.2.2
            \subsubsection{实验过程}
            用户自传播又称作裂变传播,是整个AARRR模型的最后一环。实现用户的裂变
            传播,才能带来用户的低成本增长,实现用户群体的爆发式增长。

            如果社群的内容或者活动本身就具备可玩性,或者用户的某种需求点被触动,
            用户也会自发的去将内容传播到自己的社交圈子,带来新的用户。而达成这一
            点,社群的运营也就形成了一个闭环。

    % 4
    \section{实验结果}
    社群运营不但能够解决目标的流量瓶颈问题,还能够对用户进行深度运营。建立一个良
    好的高价值的群,是商家盈利的重要因素,每个社群成员都有裂变出更多客户的可能,
    而建立完善的用户运营模型,能够有效的管理社群成员,实现用户的裂变增长。

    % 5
    \section{结论}

    % 6
    \section{致谢}

\clearpage
\end{CJK}
\end{document}
